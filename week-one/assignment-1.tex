\documentclass[11pt]{article}
% Essential packages for precise mathematical typesetting
\usepackage[margin=1in]{geometry}
\usepackage{amsmath, amssymb, amsthm}  % Fundamental mathematical notation
\usepackage{enumitem}                  % For structured enumeration
% Theorem environments (though not needed here, included for completeness)
\newtheorem{theorem}{Theorem}
\newtheorem{definition}{Definition}
% Mathematical set notation (for potential use in rigorous analysis)
\newcommand{\R}{\mathbb{R}}
\newcommand{\N}{\mathbb{N}}
\newcommand{\Z}{\mathbb{Z}}
\newcommand{\Q}{\mathbb{Q}}
\title{Introduction to Mathematical Thinking\\
       Assignment 1}
\author{Ibad Desmukh\\[1ex]
        \small Course Instructor: Keith Devlin\\
        \small Stanford University - Coursera}
\date{\today}
\begin{document}
\maketitle
% For text solutions (like Assignment 1)
\section*{Solutions}
\begin{enumerate}
    \item The man looked at the woman through a telescope. The woman looked at the man through a telescope.
    \item
        \begin{enumerate}[label=(\alph*)]
            \item Sisters reunited in checkout line at Safeway after ten years.
            \item City authorities are looking into appearance of large hole in High Street.
            \item Mayor says bus passengers should keep seat belts on.
        \end{enumerate}
    \item A head injury should never be ignored.
    \item Use the stairs.
    \item The sentence is a true statement. The purpose is to convey that it is not the end of the document, or a statement in itself. Reformulation: This page intentionally left blank. This is not the end of the document. 
    \item St Pancras plans for direct trains from UK to Germany.  Families return home after sinkhole swallows road. 
    \item The sentence assigns a qualitative value i.e. hot, to a quantitative variable i.e. temperature. Hot could mean different values of temperature in different contexts. This could be disastrous in situations where temperature needs to be carefully monitored with high accuracy.
    \item We begin by searching
        \begin{itemize}
            \item $(2 \times 3) + 1 = 7$
            \item $(2 \times 3 \times 5) + 1 = 31$
            \item $(2 \times 3 \times 5 \times 7) + 1 = 211$
            \item $(2 \times 3 \times 5 \times 7 \times 13) + 1 = 30,031$
        \end{itemize}
\end{enumerate}
\subsection{Just for fun}
\begin{enumerate}
    \item The context is someone explaining their position on a subject to another person. The person explaining their position pauses between each part of the explanation. Sentence: `Why do you think that is the case? And? And? And? And? And? Ok, that makes sense.'
    \item The context is that a person is explaining a technical concept to another person. Sentence: `When a ball is released, gravity causes it to move down and, or, and, or, touch the ground and bounce back from it.'
\end{enumerate}
\end{document}